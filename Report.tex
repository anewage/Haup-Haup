\documentclass{report}
\usepackage[table,xcdraw]{xcolor}
\usepackage[utf8]{inputenc}
\usepackage{geometry}
\usepackage{amsmath}
\usepackage{amsthm}
\usepackage{amsfonts}
\usepackage{amssymb}
\usepackage{graphicx}
\usepackage{tocloft}
\usepackage{pgfplots}
\usepackage{multirow}	
\usepackage{tikz}

\usepackage[usenames,dvipsnames]{pstricks}
\usepackage{epsfig}
\usepackage{pst-grad} % For gradients
\usepackage{pst-plot} % For axes
\usepackage[space]{grffile} % For spaces in paths
\usepackage{etoolbox} % For spaces in paths

%\usepackage{perpage} %the perpage package
%\MakePerPage{footnote} %the perpage package command

% For adding TOC in pdf bookmarks
\usepackage{hyperref}
\hypersetup{
	colorlinks=true,
	linkcolor=blue,
	filecolor=magenta,      
	urlcolor=cyan,
}
\usepackage{hypcap}
%
\usepackage{float}
\usepackage{listings}
\usepackage{color}

\definecolor{codegreen}{rgb}{0,0.6,0}
\definecolor{codegray}{rgb}{0.5,0.5,0.5}
\definecolor{codepurple}{rgb}{0.58,0,0.82}
\definecolor{backcolour}{rgb}{0.95,0.95,0.92}

\lstdefinestyle{mystyle}{
	backgroundcolor=\color{backcolour},   
	commentstyle=\color{codegreen},
	keywordstyle=\color{magenta},
	numberstyle=\tiny\color{codegray},
	stringstyle=\color{codepurple},
	basicstyle=\footnotesize,
	breakatwhitespace=false,         
	breaklines=true,                 
	captionpos=b,                    
	keepspaces=true,                 
	numbers=left,                    
	numbersep=5pt,                  
	showspaces=false,                
	showstringspaces=false,
	showtabs=false,                  
	tabsize=2
}

\lstset{style=mystyle}
\usepackage{xepersian}
\usepackage{bidi}

\settextfont{Yas}
\SepMark{-}

\renewcommand{\cftsecleader}{\cftdotfill{\cftdotsep}}

\theoremstyle{definition}
\newtheorem{definition}{تعریف}

\title{تمرین درس طراحی کامپایلرها}
\author{امیر حقیقتی ملکی}
\date{پاییز ۹۶}
	
\begin{document}
	%%%%%%%%%%%%%%%%%%%%%%%%%%%%%%%
	%%	 TITLE PAGE - BEGIN	     %%
	%%%%%%%%%%%%%%%%%%%%%%%%%%%%%%%
	\newgeometry{margin=1in}
	\pagenumbering{gobble}
		\begin{titlepage}
		\centering
		\includegraphics[width=0.25\textwidth]{../../../Template/Resources/logo.png}\par\vspace{1cm}
		{\scshape\LARGE دانشگاه صنعتی امیرکبیر \par}
		{\scshape\LARGE دانشکده مهندسی کامپیوتر و فناوری اطلاعات \par}
		\vspace{1cm}
		{\scshape\Large
	گزارش پروژه
			\par}
		\vspace{1.5cm}
		{\huge\bfseries 
			پیاده‌سازی اتوماتای محدود برای تحلیل‌گر نحوی
			\par}
		\vspace{2cm}
		{\Large امیر حقیقتی ملکی \par}
		{\Large ۹۳۳۱۰۰۹\par}
		\vfill
		استاد درس:\par
		دکتر ممتازی
		\vfill
		
		% Bottom of the page
		{\large \rl{
				زمستان ۹۶
			}\par}
	\end{titlepage}
%	\newpage
%	\pagenumbering{gobble}
	%\tableofcontents
	\newpage
	\pagenumbering{arabic}
	برای پیاده‌سازی از روش مستقیم استفاده شده که برای دادن گرامر به این برنامه می‌بایست در زمان‌اجرا و در کنسول، خطر به خط گرامر را وارد کنیم.
	
	همچنین شایان ذکر است که می‌بایست عبارت‌های پایانی\footnote{non-Terminals} حتما درون علامت دو گیومه ("") قرار بگیرند و میان هر سمبل حداقل یک فاصله بایستی باشد. همچنین در پایان حتما می‌بایست واژه end ذکر شود.
	نمونه گرامر در زیر ذکر شده:
	
	\begin{latin}
		\begin{lstlisting}[language=bash]
E : T Y
T : "int"
T : "(" E ")"
Y : "+" E
Y : "epsilon"
end
\end{lstlisting}
	\end{latin}
\end{document}